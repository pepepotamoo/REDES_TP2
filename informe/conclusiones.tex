\section{Conclusiones}
Al haber realizado las experimentaciones pudimos destacar varias cosas:
\begin{itemize}
\item Los enlaces submarinos son detectados por los valores más altos en los $\Delta RTT$, como se vio en cada experimento. Si analizamos la teoría del test de grubbs, justamente se quedaba con los valores más altos, aunque no sabíamos de que magnitud deberían ser estos para tomarlos en cuenta como enlaces submarinos. Sin embargo vemos que funciona bien en la experimentación y en general capturó los Hops necesarios para encontrar las rutas significativas en la traza.
\item La herramienta de geolocalización que usamos para los primeros dos casos no fue muy eficiente. Dio información erronea, localizando ip que está en un país cuando en realidad está en otro, pudiendo ésta estar en otro continente. Los errores fueron de uno o dos saltos de diferencia, y sucedía en general cuando había algún cambio de país alrededor de esos saltos.\\
La herramienta utilizada en el último experimento dió información bastante exacta, aunque creemos que puede producir errores para otros experimentos.\\
En conclusión, con las herramientas de geolocalización podemos tener una idea del recorrido del paquete hacia el destino, pero no podemos tener certeza sobre una ip en particular.
\item Notamos en los tres experimentos que, a medida que nos vamos alejando salto a salto, los tiempos en el grafico del RTT promedio van siendo cada vez menos constantes, y en los de $\Delta RTT$ generan algunos picos que llaman la atención, aunque estén dentro del mismo país. Incluso, en el primer experimento, nos llego a detectar un outlier falso por dicha variación, ya que en el Hop 16 genera un pico descendente importante, aunque creemos que sobre este Hop inciden muchos otros factores como lo explicamos en su análisis correspondiente.
\item Las tres experimentaciones partieron de Argentina, pasaron por Estados Unidos y de ahi fueron a su país destino. En ninguno de los tres experimentos parece el recorrido más eficiente, dado que en el primero y tercer experimento podría haber pasado por algún país de Africa y de ahi pasar a Europa, y en el segundo experimento ir directo a Sidney desde Buenos Aires. Creemos que esto pudo deberse a que los tres usamos el mismo proveedor de internet para la experimentación, Fibertel.
\item Los time out que generan algunos routers y su forma de resolverlos en la implementación, es importante tenerla en cuenta, ya que nos modifica a la hora de encontrar outliers.\\ 
Hay muchas formas de resolverlo, la nuestra, como explicamos, fue la de interpolar el RTT del Hop antecesor al que se generó el time out, con el Hop sucesor. De esta manera logramos obtener el RTT de este paquete que no lo supimos por un time out originado. Esta solución nos trajo un problema que pudimos observarlo en la experimentación. Si cae un time out en un Hop $i$ y el Hop $i+1$ marca una distancia muy grande con el Hop $i-1$, entonces $i+1$ es una gran candidato a ser un enlace submarino; pero al colocarle al Hop $i$ un RTT con el promedio de $i+1$ e $i-1$, la diferencia grande de tiempo que se podría haber observado se reduce a la mitad. De esta manera, puede suceder que el test de grubbs no pueda detectar este enlace submarino, cuando en realidad lo es. Efectivamente esto se comprueba en el experimento 1, al saltar de Inglaterra a Rusia, que tiene una distancia bastante grande y sin embargo no es detectado por el test.
\end{itemize}